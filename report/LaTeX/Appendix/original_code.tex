\definecolor{dkgreen}{rgb}{0,0.6,0}
\definecolor{gray}{rgb}{0.5,0.5,0.5}
\definecolor{mauve}{rgb}{0.58,0,0.82}

\lstset{frame=tb,
  language=C,
  aboveskip=3mm,
  belowskip=3mm,
  showstringspaces=false,
  columns=flexible,
  basicstyle={\small\ttfamily},
  numbers=left,
  numberstyle=\tiny\color{gray},
  keywordstyle=\color{blue},
  commentstyle=\color{dkgreen},
  stringstyle=\color{mauve},
  breaklines=true,
  breakatwhitespace=true,
  tabsize=3
}
\begin{lstlisting}
// original code fragment:

#include <stdio.h>;
#include <ctype.h>;
#include <string.h>;
#include <stdlib.h>;

void func1() {	
    char buffer[1024];
    
    printf("Please enter your user id :");
    fgets(buffer, 1024, stdin);
     
    if (!isalpha(buffer[0])) {
        char errormsg[1044];
             
        strncpy(errormsg, buffer, 1024);
        strcat(errormsg, " is not  a valid ID");
 	}
}

void func2(int f2d) {
    char *buf2;
    size_t len;

    read(f2d, &len, sizeof(len));
    buf = malloc(len + 1);
    read(f2d, buf2, len);
    buf2[len] = '\0';
}

void func3(int f3d){
    char *buf3;
    int i, len;
         
    read(f3d, &len, sizeof(len));
         
    if (len > 8000) { 
        error("too large length"); 
        return; 
    }

    buf3 = malloc(len);
    read(f3d, buf3, len);    
}

void main()
    char *foo = "fooooooooooooooooooooooooooooooooooooooooooooooooooo";
    char *buffer = (char *)malloc(10 * sizeof(char));
    
    strcpy(buffer, foo);
    func1();
    func3(len(*foo));
}
\end{lstlisting}
